% THIS IS A LATEX TEMPLATE FILE FOR PAPERS INCLUDED IN THE
% *Anthology of Computers and the Humanities*. ADD THE OPTION
% 'final' WHEN CREATING THE FINAL VERSION OF THE PAPER. 
% DO NOT change the documentclass
%\documentclass[final]{anthology-ch} % for the final version
\documentclass{anthology-ch}         % for the submission

% LOAD LaTeX PACKAGES
\usepackage{booktabs}
\usepackage{graphicx}
% ADD your own packages using \usepackage{}

% TITLE OF THE SUBMISSION
% Change this to the name of your submission
\title{Unstable Data and the Unusual Case of the Prosody Excerpt in the Digital Library}

% AUTHOR AND AFFILIATION INFORMATION
% For each author, include a new call to the \author command, with
% the numbers in brackets indicating the associated affiliations 
% (next section) and ORCID-ID for each author.  
\author[1]{Rebecca Sutton Koeser}[
  orcid=0000-0002-8762-8057
]

\author[1]{Mary Naydan}[
  orcid=0000-0002-7960-3175
]

% While we encourage including ORCID-IDs for all authors, you can
% include authors that do not have one by definining an empty ID.
%\author[1,2]{Meredith Martin}[
\author[1]{Meredith Martin}[
  orcid=0000-0003-0214-8757
]

% There should be one call to \affiliation for each affiliation of
% the authors. Multiple affiliations can be given to each author
% and an affiliation can be given to multiple authors. 
\affiliation{1}{Center for Digital Humanities, Princeton University, Princeton, New Jersey, USA}
%\affiliation{2}{English Department, Princeton University, Princeton, New Jersey, USA}

% KEYWORDS
% Provide one or more keywords or key phrases seperated by commas
% using the following command
\keywords{humanities data, unstable data, reproducibility, digital libraries}

% METADATA FOR THE PUBLICATION
% This will be filled in when the document is published; the values can
% be kept as their defaults when the file is submitted
\pubyear{2025}
\pubvolume{1}
\pagestart{1}
\pageend{1}
\conferencename{Proceedings of Conference XXX}
\conferenceeditors{Editor1 Editor2}
\doi{00000/00000}  

\addbibresource{bibliography.bib}

%%%%%%%%%%%%%%%%%%%%%%%%%%%%%%%%%%%%%%%%%%%%%%%%%%%%%%%%%%%%%%%%%%%%%%%%%%%
% HERE IS THE START OF THE TEXT
\begin{document}

\maketitle

%TC:ignore
\begin{abstract}
This paper examines the challenges of unstable data in computational humanities research using the case study of the Princeton Prosody Archive (PPA) and HathiTrust Digital Library. After introducing both resources, we discuss the unusual way the PPA dynamically pulls content from multiple HathiTrust systems, which makes excerpts (book chapters and journal articles) particularly susceptible to changes from HathiTrust updates. We quantify the estimated rate of change in PPA excerpts, in the PPA at large, and across all of HathiTrust. We address the implications of these findings for reproducibility and for building on previous scholarship. Ultimately, we aim to raise awareness about the kinds of hidden changes that occur in large-scale cultural heritage collections, which our field necessarily relies upon as data for research.  
\end{abstract}
%TC:endignore

\section{Introduction}

Data is essential for computational humanities research, but humanities data is rarely static. Whether research is based on small-scale data to test a particular method, or expansive collections for measuring larger trends across huge swathes of content or time, or mid-size data somewhere in between, understanding the contents and provenance of a dataset is crucial to interpreting the results. The ability to share or recreate a dataset is equally crucial for reproducible research and to assess, critique, and take up new methodologies. However, the ability to build on previous research is difficult when there is instability in research data sources, particularly when that instability is unexpected or incompletely understood. 

Unstable data is particularly challenging when data is used in novel or creative ways that the data curators or publishers did not anticipate. Rufus Pollock, an early proponent of open knowledge and open data, has long argued that “the best thing to do with your data will be thought of by someone else”  \cite{pollock_open_2011}. New and creative uses of data, and connections between data, are essential to research and new discoveries, but computational humanists’ reliance on data from the GLAM (Galleries, Libraries, Archives, and Museums) sector poses particular challenges. In this paper, we share our experience working with unstable data in novel ways using the Princeton Prosody Archive (PPA) as a case study, with a particular focus on HathiTrust Digital Library. We use this example to explore the ways humanities data tests the limits of current guidelines and recommendations for reproducible research and humanities data publication, and we consider the implications for the field of computational humanities. 

Reproducible data for the humanities is a longstanding issue. The \textit{Journal of Open Humanities Data} celebrates its 10th anniversary this year, and while it publishes articles about humanities research objects or techniques “with high potential for reuse” \cite{noauthor_journal_2025}, reproducible data is not required.  The \textit{International Journal of DH} published a special issue on “Reproducibility and Explainability” in 2023 \cite{ries_reproducibility_2024}, inspired by Nan Z. Da's well-known critique, “The Computational Case against Computational Literary Studies” \cite{da_computational_2019}. In their introduction to the special issue, Ries et al. are clear that reproducibility issues impact every scientific field. As we know from other research, the adoption of machine learning methods and data- and model-driven analysis has exacerbated the problem \cite{kapoor_leakage_2023}.  Addressing historical research projects in particular, Toby Burrows makes the case for robust documentation of versions, dates, formats, and other factors when creating and using digital collections for historical research, since the “changeability and instability of digital collections are usually considerably greater” than analog archives \cite{burrows_reproducibility_2023}. After all, if humanities collections become data, then the need to cite a stable humanities data source is as important as citing the proper location in a book or volume. The Collections as Data community has sought to collaboratively develop cultural heritage collections in ways that support computational research \cite{padilla_collections_2023, padilla_final_2019}, since scholars across the humanities and information sciences are not always aligned. For instance, Melanie Walsh’s position statement on Collections as Data describes her “dream” of “authority records for works,” since humanities research questions often don't need “all the different editions and variations of a book” \cite{chambers_position_2023}, but this is at odds with the way that the Library and Information Science community configures collections around specific editions.\footnote{The problem of multiple editions of the same work also came up within the Princeton Prosody Archive; see a 2019 essay by Wilson and Naydan on “Deduplicating the Archive” for more details. \cite{wilson_deduplicating_2019}} 

\section{Unstable data in the Princeton Prosody Archive}

\subsection{Princeton Prosody Archive }

The Princeton Prosody Archive (PPA) is an open-source, full-text searchable database of 6,000+ English-language digitized works about the study of poetry, versification and pronunciation \cite{noauthor_princeton_nodate}. The works in the PPA — which include grammar books, elocution manuals, and scholarly articles, among many other kinds of books — were all published between 1532 and 1928, the current cut-off year for works in the public domain in the United States.

The PPA is implemented as a custom Python/Django web application built by the Center for Digital Humanities at Princeton under the technical leadership of Rebecca Sutton Koeser and co-PIs Meredith Martin and Mary Naydan. The PPA initially supported only HathiTrust content and full works, but the data model was intentionally simple so it could be expanded to include other content. Now, the PPA also includes content from Gale/Cengage’s Eighteenth Century Collections Online (ECCO), as well as Early English Books Online using editions from the Text Creation Partnership (EEBO-TCP). The PPA also supports excerpted works, such as book chapters or journal articles, so that only the relevant portions of larger works are included. There is no other scholarly resource in the humanities that provides access to a curated selection of works from larger proprietary databases in one place. 

\subsection{HathiTrust Digital Library}

The majority of the works in the PPA come from HathiTrust Digital Library, a US-based, not-for-profit consortium of over 60 academic and research libraries from across North America and other countries. HathiTrust Digital Library began in 2008 with Google’s mass digitization initiative and now provides “reading access” to its 18+ million digitized volumes “to the fullest extent allowable by U.S. and international copyright law” \cite{noauthor_welcome_nodate}.

Similar large-scale aggregators exist in Europe and beyond, though these are often led by national libraries. For instance, \textbf{TROVE}, maintained by the National Library of Australia, collects billions of digital items from Australian libraries, universities, museums, galleries, and archives \cite{noauthor_home_nodate}. \textbf{Gallica}, the digital library of the Bibliothèque nationale de France (BnF) and over 300 partners, provides access to over 10 million items \cite{noauthor_page_nodate}. \textbf{Europeana} makes available various kinds of media and accompanying metadata from thousands of cultural institutions \cite{noauthor_discover_nodate}. Smaller cultural heritage aggregators include Switzerland’s \textbf{e-rara} \cite{noauthor_e-rara_nodate} and \textbf{e-manuscripta} \cite{noauthor_e-manuscripta_nodate} and Finland’s \textbf{Finna} \cite{noauthor_search_nodate}.
 
We mention these aggregators to suggest that the kinds of data instability we encountered in HathiTrust (explained further below) will be found in \textit{any }large-scale cultural heritage aggregator because of GLAM workflows to manage and improve collections, as well as the ephemeral nature of technical infrastructure. These changes often involve a trade-off between reproducibility and improvement. For example, TROVE’s crowdsourced text-correcting feature improves OCR transcripts, but means that researchers will be working with different transcripts depending on the date of download \cite{noauthor_text_nodate}. This opportunity for continued enhancement of the content is valuable, but depending on the extent of the changes, could have downstream effects on any word-level analysis. As another familiar example, Europeana discusses the challenge of finding and fixing broken links and the reality of not being able to fix some material \cite{noauthor_keeping_nodate}.

\subsection{PPA’s unusual use of HathiTrust}

While the constant revisions or improvement of data would affect any scholar working with GLAM collections, the PPA’s highly unusual use of HathiTrust’s data made this encounter unusually visible. The PPA uses HathiTrust’s bibliographic API to import bibliographic metadata, which it presents alongside page-level data (OCR) from local copies of METS XML and plain text files stored via rsync and page image thumbnails dynamically loaded from HathiTrust’s image server. While HathiTrust supported our work through several rounds of negotiation and renegotiation, their infrastructure was not designed to be used by external projects in this persistent, dynamic manner. Nevertheless, we had been using content from HathiTrust in PPA for years before we truly understood the degree of change and instability in the data. However, there were plenty of indicators along the way. In retrospect, HathiTrust’s initial pushback on our request for on-going rsync access to full-text content for the records in our dataset was one, since rsync was typically used to provide access to a one-time snapshot of volumes in a requested dataset. Our request for page image thumbnails — essential for seeing the unique and elaborate prosodic markings indicating the pronunciation of poetic lines — received a different kind of pushback due to Google’s copyright of the image scans. A similar indication was the emails HathiTrust regularly sends out with lists of records that are no longer public domain, which dataset users must remove from their copy of HathiTrust data (see Appendix \ref{appdx:first}, Appendix \ref{appdx:second}). Since PPA contains only a tiny fraction of HathiTrust’s 18+ million volumes (Figure \ref{fig:bubble-chart}), and data curation prioritized works from Princeton and affiliated libraries, this has not had a noticeable impact on our data.

%TC:ignore
\begin{figure}
    \centering
    \includegraphics[width=0.75\linewidth]{figures/bubble-chart.png}
    \caption{This bubble chart, originally published in Rebecca Sutton Koeser’s essay “Visualizing the Collections,”\cite{koeser_visualizing_2020} shows the relative scale of HathiTrust and PPA as of 2020. At the time of writing, the PPA contains 5,480 HathiTrust works, or 0.025\% of HathiTrust.}
    \label{fig:bubble-chart}
\end{figure}
%TC:endignore

This vast difference in scale is actually a reason for the PPA’s creation; such a specialized discourse would easily get lost when searching the full scale of HathiTrust’s collection. Other, more standard collaborations with HathiTrust than ours share this same motivation. Many HathiTrust-based projects focused on various methods for, first, finding certain kinds of material within HathiTrust’s massive digital library, and then testing various text and data mining methods on that collection using the tools and services provided by HathiTrust Research Center (HTRC)\cite{noauthor_hathitrust_nodate}. Prominent examples of this two-step process include Gioia Stevens’s \textbf{Early American Cookbooks} project (now a permanent collection within HathiTrust) \cite{stevens_new_2017} and Laure Thompson and David Mimno’s \textbf{20th Century English-Language Speculative Fiction }(a recommended workset within HTRC) \cite{thompson_building_nodate, 
noauthor_recommended_nodate}. Recent work from Ryan Dubnicek and collaborators has focused on using machine learning methods to discover particular genres within HathiTrust Digital Library \cite{parulian_uncovering_2022, dubnicek_ryan_piloting_2023} and building out the customizable capacities of HTRC’s Extracted Features dataset through the development of an open API \cite{john_a_walsh_library_nodate}. While this work has been important, the algorithms, datasets, and advanced computing environments are all contained within HTRC’s “walled garden,” a closed platform with controlled access and limited interactions. Part of what makes the case of the PPA so unusual is that it brings HathiTrust’s data out of the “walled garden” and puts it in conversation with works that are not found in HathiTrust. We believe that working across walled gardens is the future of computational scholarship, as evidenced by the sunsetting of JSTOR’s TDM platform Constellate on July 1, 2025 \cite{noauthor_constellate_2019} and HathiTrust’s announcement in October 2024 that it is shutting down the HTRC because “many of our members have rarely utilized HTRC’s services,” leaving the future of TDM on HathiTrust works uncertain \cite{noauthor_plans_nodate}.

Cutting across walled gardens to incorporate works from Gale/Cengage’s ECCO into the PPA introduced different kinds of instability. Indeed, the ECCO data provides an interesting counterpoint to the instability of HathiTrust: the data itself is unchanged, and the OCR text for these materials have not been updated since 2008, even though there have been substantial improvements in OCR technology since the collection was originally digitized, in spite of the known poor quality of the text \cite{hill_quantifying_2019}. In this case, we encountered instability in the technical and human infrastructure. In the time since we first added the integration in 2021, Gale/Cengage made API changes that broke PPA links to image thumbnails and links to ECCO for non-Princeton users. These changes required work on our part to adapt our systems and fix the issues. As with HathiTrust, we have also encountered personnel turnover, where our key contacts and collaborators have left without notice. Although the PPA's agreement with each institution is formalized via a written MOU (Memorandum of Understanding), our collaborators generally view our project as an exception or experiment for an unusual use case, so there is no plan for continuity when our contacts leave, as Koeser, Naydan, and Martin have previously discussed \cite{naydan_beyond_2024}.

It is difficult to identify other digital projects with unusual uses of data or technical infrastructure from an outside perspective; this requires insider knowledge of the process of working with vendors to build projects, which is less often narrated than research methods, analytical results, or final outputs. Meredith Martin’s new monograph, \textit{Poetry’s Data}, is a notable exception to this trend and argues for the value of narrating the process of navigating our digital research infrastructures \cite{martin_poetrys_2025}. The Princeton Geniza Project and the Shakespeare \& Company Project are two additional examples of unexpected use: both projects attach research data to digital images published by GLAM institutions via International Image Interoperability Framework (IIIF) APIs \cite{noauthor_princeton_nodate, noauthor_shakespeare_2020}. This kind of persistent use is within the guidelines of the published IIIF APIs, but atypical.

\subsubsection{The problem of excerpts}

The most unusual aspect of PPA’s use of HathiTrust is arguably its support for book excerpts and journal articles. HathiTrust scans and indexes full works and full volumes of journals; it does not provide any consistent, reliable, or systematic semantic structure to identify the chapters or articles contained within them. After the PPA's initial launch in 2019, however, it quickly became clear that excerpt-level indexing would surface discoveries that would otherwise remain buried beneath unspecific metadata \cite{naydan_book_2024}. Between 2020 and 2021, the PPA project team identified relevant journal articles and book chapters within larger works and curated the excerpt-level metadata manually. 

Integrating excerpts initially surfaced the unstable data issue, as the team encountered mismatches between the thumbnail image and full-text snippets of excerpts after implementing the functionality. Our previously correct page ranges were now wrong. How was this possible? Because of the way we were dynamically pulling content from multiple HathiTrust systems, we discovered that the digital page ranges of content could shift whenever HathiTrust rescanned works in ways that altered the digital page numeration.

We suspect that HathiTrust’s regular rescanning of material is not widely known or even fully understood among researchers. The only public indication is the following disclaimer on an “About” page (emphasis added): "\textit{The HathiTrust collection is not static.} Works get added to the collection every day, and \textit{sometimes a digital item may be updated with a new version}. Bibliographic records can be updated when contributors send us corrections. Copyright and access statuses may change as items undergo copyright review or we receive permissions agreements from copyright holders" \cite{noauthor_how_nodate}. HathiTrust’s “Full view” page for individual items also provides clues to changes. A  “Version” label with a date indicates when an item was last updated. However, this date could indicate a substantive change, like a full or more complete rescanning; a less obvious change, such as the addition of deep page-level metadata; or a trivial change, such as metadata corrections. There is no way to easily identify the type of change.\footnote{In our investigations, we discovered that the modification time of the METS XML file corresponds to the date in the public version label. The text content could be modified before or after this time. }

The project team brainstormed ways to automate fixing range changes for excerpts whenever HathiTrust made updates to works in the PPA. At first glance, it seems like a task that should be computationally tractable, and when we presented this unsolved problem to a junior computer science independent work seminar taught by Brian Kernighan, Kernighan thought so, too. His investigations usefully illustrate the problem we faced (Figure  \ref{fig:code-comparison}).

%TC:ignore
\begin{figure}
    \centering
    \includegraphics[width=0.5\linewidth]{figures/y.png}
    \caption{Side-by-side screenshots showing page-level metadata for “The Scansion of Middle English Alliterative Verse” by William E. Leonard \cite{noauthor_scansion_nodate}, generated from the page’s HTML and javascript, on two different dates: March 18, 2024 (left) and March 26, 2024 (right). Some of the changes involve recharacterizing page-level semantic metadata (e.g. flagging “UNTYPICAL\_PAGE” or section starts, though it is worth noting that these changes are not always more accurate). Note also the change in purported page numbers (“pageNum”) beginning with sequence (“seq”) 134. The OwnerID number, used in the permanent link to page scans, also changes relative to the accompanying sequence and page numbers.}
    \label{fig:code-comparison}
\end{figure}
%TC:endignore

Digging deeper, Kernighan discovered especially thorny cases, such as Figure \ref{fig:MLN}.

%TC:ignore
\begin{figure}
    \centering
    \includegraphics[width=1\linewidth]{figures/Screenshot 2025-07-10 at 3.19.38 PM.png}
    \caption{The above page from O. F. Emerson’s “The Development of Blank Verse: A Study of Surrey” in the journal \textit{Modern Language Notes} \cite{emerson_o_f_development_1889} displays \textit{three} different page numbers. Each column of \textit{MLN }gets its own page number (left, 467; right, 468), and each page gets a third, found at the bottom center (234). HathiTrust’s black navigation controls show that it takes the upper left page number as the original/physical page number (p. 467). This means that scrolling through HathiTrust’s reader, the original page numbers increase by two instead of one. The digital page number/sequence number (\#296) is based on scanning and subject to change with rescanning.}
    \label{fig:MLN}
\end{figure}
%TC:endignore

It became apparent that automating page range fixes would not be trivial due to cases like these. Although there are certainly approaches that could help, our proposed solution involved flagging recently changed volumes for manual review. However, the ongoing human curation needed, in addition to the technical complexity, led us to decide that implementation was out of scope for that project.

\section{Quantifying HathiTrust’s rate of change}

\subsection{Changes in PPA excerpts}

A spreadsheet from November 2023 to correct the page ranges for PPA excerpts from HathiTrust provides a window into the degree of instability and changes we encountered over roughly two years. After we discovered that we were pulling erroneous content for some excerpts, we tasked a student researcher with manually checking and correcting the digital page range for excerpts that had shifted. Of 517 total excerpts, 121 excerpts (23.4\%) had changed. For 10 of those excerpts, the number of pages included changed; in these cases, the most common change was 2 pages.\footnote{There were two outliers with differences in length of 25 and 16 pages; preliminary investigations suggest that these were due to data errors in the original excerpts. For instance, one excerpt was originally only pulling a single page when the article was actually several pages long; most likely, someone neglected to input the end range. However, it is difficult to confirm with certainty after the fact.} While unusual, this might occur if, for instance, HathiTrust erroneously scanned the same page twice within an excerpt sequence, and then updated the scan to rectify the error. If we look at the difference in start pages between these updates, we find an average shift of 6.4 pages; 9 volumes shifted by more than 10 pages, 5 by more than 10, and one outlier, Thomas Stewart Omond’s “Swinburne as a Metrician” in volume 76 (1909) of \textit{The Academy and literature} \cite{omond_thomas_stewart_swinburne_1909}, shifted by 240 pages! This could happen if HathiTrust replaced a partial scan with a more complete one. Most troubling, when we analyze the overlap between pages in range before and after these updates, \textit{there are 36 excerpts with no pages in common between the initial and updated page range}; that is, without updating the page range, we would be including \textit{none} of the intended contents.

\subsection{Changes in all of HathiTrust}

Observing this volatility at the small scale of PPA excerpts led us to wonder about the rate of change for PPA content more generally and in HathiTrust at scale. Fortunately for us, HathiTrust is incredibly transparent and publishes monthly and daily files with data about updated records. They note that a volume might be included in this update file for any of the following reasons, whether newly deposited, a new copy of an existing item, changes to rights and access permission, or an update to the bibliographic metadata \cite{noauthor_hathifiles_nodate}.  Figure \ref{fig:hathi-daily-updates} charts the number of updated volumes across all of HathiTrust from May 1 to July 10, 2025.
\begin{figure}[t!]
    \centering
    \includegraphics[width=1\linewidth]{figures/hathitrust_changes_countonly.pdf}
    \caption{Number of volumes updated daily in all of HathiTrust from May 1 to July 10, 2025.}
    \label{fig:hathi-daily-updates}
\end{figure}
The number of updates varies on any given day from nearly negligible to more than 800,000 volumes. The day with the most updates during this time period was May 15, 2025 with 846,329 volumes updated (4.5\% of all of HathiTrust). The average daily update over this period is 348,392, which is only 1.8\% of all of HathiTrust.

\subsection{Changes in all of PPA}

What about all the HathiTrust volumes within the Princeton Prosody Archive? How frequently are they changing? One way to answer this question is to filter the data on HathiTrust updates to volumes included in PPA; when we do, we find that PPA updates track somewhat with the larger updates (refer to Figure \ref{fig:ppa-daily-updates} and compare with Figure \ref{fig:hathi-daily-updates}). Although these changes are on a much smaller scale, there are still multiple days when more than a hundred volumes changed.

\begin{figure}[t!]
    \centering
    \includegraphics[width=1\linewidth]{figures/ppa_hathitrust_changes_countonly.pdf}
    \caption{Number of PPA volumes updated daily in all of HathiTrust from May 1 to July 10, 2025.}
    \label{fig:ppa-daily-updates}
\end{figure}

We happened to have another way to investigate this question: we have two different snapshots of the PPA full-text corpus which we can compare. One version includes HathiTrust content which was last updated November 2024; the other is an export created February 2025, a difference of roughly three months.\footnote{The earlier version is the corpus that has been used as the basis for computational research on PPA; the later one was generated as part of a matching text and image snapshot which was provided as a one-time dataset by HathiTrust.} There are slight differences between the number of pages in these two versions, with a little over 1.5 million pages in common. When we match pages based on exactly equal text contents within the same volume (excluding pages with no text), we find 855,930 (55.8\%) matching pages across the two corpora; of those, 14,423 pages (1.7\% of the total 1.5 million pages) have shifted sequence within their volume, indicating a change in structure with no update to the text. The mismatches for the remaining pages (44.2\% of our corpus) are likely due to OCR changes. While it is lovely to think that the content has been improved, this means that any word or token-level analysis on the prior version of the corpus would need to be either rerun or realigned to the updated text.

%TC:ignore
\begin{figure}[t!]
    \centering
    \includegraphics[width=1\linewidth]{figures/ppa_hathitrust_lastmodified.pdf}
    \caption{PPA volumes by last modification date, for volumes in PPA production and computational corpus.}
    \label{fig:ppa-last-modified}
\end{figure}
%TC:endignore

\subsection{Implications}

When PPA research efforts shifted from data aggregation, curation, and presentation to computational analysis based on the PPA text corpus, we had new opportunities to build on prior research; however, this made the repercussions of the instability of our source data much more obvious. PPA researchers identified an initial corpus-wide research project to detect and identify lines of poetry quoted across the million+ pages. Systematically identifying these quoted poems was a first step towards answering research questions about English prosody; a dataset of poem excerpts cited in the PPA could illuminate when and how particular poets or poems became the exemplars for particular poetic forms or figures of speech, or how quickly after publication a poet’s work becomes a canonical example, or the network of examples being reused from other prosodists.

As we considered possible approaches to the problem of poetry detection at scale, one collaborator suggested making use of a dataset of page-level genre predictions for HathiTrust volumes created by Ted Underwood \cite{underwood_page-level_2014}. This dataset includes page-level predictions because, as the researchers note, volumes are rarely a single genre and often include collections of disparate materials. While the genre prediction task is not strictly the same as poetry detection, there is enough overlap that data on pages predicted as poetry could be used as a starting point or confirmation of results from other methods. However, the degree of change possible in HathiTrust materials means that this page-level data is basically unusable; there may be volumes included where the pages have not changed, but determining which ones those are would be difficult. As a demonstration of this problem, we offer one example. The essay “On Stile and Versification” by William Belsham \cite{belsham_stile_1799} is one of the excerpts included in PPA with known poetry excerpts; many of the pages include short poetry quotations, and there are two pages in sequence that are all or almost all poetry. Nearly all of the pages in this volume are classified as nonfiction prose in the page-level genre dataset, except for two pages in sequence that are labeled as poetry near the page range of this excerpt. However, the page indexes don’t match any of the expected pages in our data. In Underwood’s page-level genre metadata, the poetry pages are numbered 510 and 511. (In our current volume, the digital pages are 515 and 516; the original printed pagination is 507 and 508).

%TC:ignore
\begin{figure}[hbt!]
  \centering
  \includegraphics[width=0.35\linewidth]{figures/hathi-pages/njp-32101076530979-515-1752088772.pdf}
  \includegraphics[width=0.35\linewidth]{figures/hathi-pages/njp-32101076530979-516-1752088741.pdf}
  \caption{Two pages of all or almost all poetry with page-level genre labels of poetry.}
  \label{fig:poetry_pages}
\end{figure}

\begin{figure}[hbt!]
  \centering
  \includegraphics[width=0.35\linewidth]{figures/hathi-pages/njp-32101076530979-513-1752088792.pdf}
  \includegraphics[width=0.35\linewidth]{figures/hathi-pages/njp-32101076530979-517-1752088752.pdf}
  \caption{Two pages with short poetry excerpts not labeled at the page level as poetry.} Images courtesy of HathiTrust. 
  \label{fig:poetry_exc_pages}
\end{figure}
%TC:endignore

The conclusions we draw from this example are that page-level genre predictions could be used to identify pages within PPA that are substantially poetry, which would be useful for some research questions and data curation tasks, and could provide partial information for the poetry detection task, but the updates to HathiTrust content since that dataset was published make it unusable for work on current copies of HathiTrust data.

Like Underwood, the PPA project team is using digital sequence to refer to pages in our found-poems dataset. What do we do with the fact that this dataset, like Underwood’s, refers to a snapshot of HathiTrust and the PPA at a moment in time? For researchers working with HathiTrust data, there is no stable page identifier across time. A stable page ID for HathiTrust is not feasible due to the scale of labor involved in creating and maintaining HathiTrust, since there are so many individual libraries with their own workflows feeding into the aggregator. While this lack of stable referents is therefore understandable, it nevertheless poses a problem for computational research and reproducibility that the field has yet to solve. 

\section{Conclusions}

There are many different modes of computational research on large-scale corpora like HathiTrust. The difficulty of dealing with the frequent changes made by HathiTrust and Gale/Cengage factored into our decision to shift from maintaining a dynamic database to extracting the data and analyzing the full-text corpus instead. In doing so, we are engaging in a familiar mode of computational scholarship: working with a frozen snapshot of a corpus to test a method or transform it into a different kind of data. This mode has the benefit of being versioned; you can point to the version of the dataset you used for your research, and sometimes even share it depending on permissions. However, this mode of scholarship prevents us from building on one another’s work. It is its own kind of walled garden, failing to feed back into or truly enhance collections like HathiTrust, Gallica, or TROVE. As long as we are doing computational research on bespoke datasets, our work doesn’t make it out of the garden, either for application in the GLAM sector or to advance domain research. 

When it is working well, GLAM data and research constitute a virtuous cycle: the data feeds the research, and then the research improves the data for everyone, as well as making research domain arguments. However, this loop is rarely so well-oiled. More frequently, the goals of the institution and the researcher are mismatched. For instance, the PPA’s unusual use of HathiTrust to dynamically present excerpts would benefit from a semantic division of texts into meaningful parts. For HathiTrust, that level of detail is not needed to achieve its goal of providing “reading access,” and it is not feasible at the massive scale at which they are working. Considering their scale of operation, HathiTrust and other aggregators have done remarkably well at being transparent about high-level changes to their collections and supporting the rather niche work of computational humanists; and the closures of Constellate and HTRC suggest they did this with little return on the investment. In the absence of aligned goals, perhaps the best we can do as researchers is to be equally transparent: to inform other researchers about the instabilities and limitations we discover in the collections we are working with, and to publish and share versioned data to the fullest extent possible. 

%TC:ignore
\section*{Acknowledgements}

...

% Print the biblography at the end. Keep this line after the main text of your paper, and before an appendix. 
\printbibliography

% You can include an appendix using the following command
\appendix

\section{HathiTrust Statement for Dataset Distribution} \label{appdx:first}

By my signature, I acknowledge and confirm the following:

\begin{enumerate}
    \item  I am receiving texts from the University of Michigan that are made available under an agreement between my sponsoring institution - [indicate sponsoring institution, e.g., Dartmouth College] - and Google.
    \item  I have read this agreement and agree to abide by its terms and to use the texts in accordance with the statement of my research, as submitted to the University of Michigan.
    \item  I agree to notify the University of Michigan of any changes that are made in the scope or nature of my research.
    \item  I understand that volumes I receive from the University of Michigan may be determined at a later date to be in copyright. I agree to delete these volumes and any copies that have been made upon notification from the University of Michigan. I agree to notify the University of Michigan at feedback@issues.hathitrust.org to confirm deletion of any such volumes.
\end{enumerate}

\rule{\textwidth}{0.5pt}

Name \hspace{0.3\textwidth} Signature \hspace{0.3\textwidth} Date

\rule{\textwidth}{0.5pt}

Title

\rule{\textwidth}{0.5pt}

Email \hspace{0.3\textwidth}  Phone


\section{Example HathiTrust deletion email for public domain dataset} \label{appdx:second}

\begin{verbatim}
Subject: Delete notifications for ht_text_pd dataset
From: HathiTrust <support@hathitrust.org>

Dear HathiTrust dataset recipient,

This email is to notify you that volumes in the HathiTrust
"ht_text_pd" dataset, of which you have downloaded all
or a subset of files, no longer meet the criteria for
inclusion in the dataset, and you no longer are allowed to
use them in your research.

Please review the data you have synced from HathiTrust to
check whether you have the volumes listed below. If so,
delete all copies you retain of these volumes in
accordance with our terms of use. Alternatively, you may
delete your copy of the dataset and re-sync to the updated
dataset.

If you no longer possess HathiTrust datasets, or if you
have other questions regarding datasets, then please email
support@hathitrust.org.

Thank you,

HathiTrust

===BEGIN ID LIST===
[ids omitted]
...
...
...
===END ID LIST===
\end{verbatim}

%TC:endignore
\end{document}
