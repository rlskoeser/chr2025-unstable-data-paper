% THIS IS A LATEX TEMPLATE FILE FOR PAPERS INCLUDED IN THE
% *Anthology of Computers and the Humanities*. ADD THE OPTION
% 'final' WHEN CREATING THE FINAL VERSION OF THE PAPER. 
% DO NOT change the documentclass
%\documentclass[final]{anthology-ch} % for the final version
\documentclass{anthology-ch}         % for the submission

% LOAD LaTeX PACKAGES
\usepackage{booktabs}
\usepackage{graphicx}
% ADD your own packages using \usepackage{}

% TITLE OF THE SUBMISSION
% Change this to the name of your submission
\title{Unstable data, Unanticipated uses}

% AUTHOR AND AFFILIATION INFORMATION
% For each author, include a new call to the \author command, with
% the numbers in brackets indicating the associated affiliations 
% (next section) and ORCID-ID for each author.  
\author[1]{Rebecca Sutton Koeser}[
  orcid=0000-0002-8762-8057
]

\author[1]{Mary Naydan}[
  orcid=0000-0002-7960-3175
]

% While we encourage including ORCID-IDs for all authors, you can
% include authors that do not have one by definining an empty ID.
%\author[1,2]{Meredith Martin}[
\author[1,2]{Meredith Martin}[
  orcid=0000-0003-0214-8757
]

% There should be one call to \affiliation for each affiliation of
% the authors. Multiple affiliations can be given to each author
% and an affiliation can be given to multiple authors. 
\affiliation{1}{Center for Digital Humanities, Princeton University, Princeton, New Jersey, USA}
% \affiliation{2}{Another Department, Another University, Another City, Another Country}

% KEYWORDS
% Provide one or more keywords or key phrases seperated by commas
% using the following command
\keywords{humanities data, unstable data, reproducibility}

% METADATA FOR THE PUBLICATION
% This will be filled in when the document is published; the values can
% be kept as their defaults when the file is submitted
\pubyear{2025}
\pubvolume{1}
\pagestart{1}
\pageend{1}
\conferencename{Proceedings of Conference XXX}
\conferenceeditors{Editor1 Editor2}
\doi{00000/00000}  

\addbibresource{bibliography.bib}

%%%%%%%%%%%%%%%%%%%%%%%%%%%%%%%%%%%%%%%%%%%%%%%%%%%%%%%%%%%%%%%%%%%%%%%%%%%
% HERE IS THE START OF THE TEXT
\begin{document}

\maketitle

\begin{abstract}
This LaTeX template helps you typeset and format a paper for the Computational Humanities Research conference in the ACH Anthology. This template helps you adhere to the the required specifications and provides an example of how your paper should look. In practice, the abstract of the paper here should be a one-paragraph summary of the outline and main contributions of the paper. 
\end{abstract}

\section{Introduction} 

Data is essential for computational humanities research, but humanities data is rarely static. Whether research is based on small-scale, curated, and annotated data to test a particular method, or large-scale and expansive collections for measuring larger trends across huge swathes of content or time, or mid-size data somewhere in between, understanding the contents and provenance of a dataset are crucial to interpreting the results. The ability to share or recreate a dataset is crucial for reproducible research, which is essential for assessment, critique, and uptake of new methodologies. However, the ability to build on previous research is difficult when there is instability in research data sources,  particularly when that instability is unexpected or incompletely understood. 

Unstable data is particularly challenging when data is used in novel or creative ways which were not anticipated by the data curators or publishers. Rufus Pollock, an early proponent of open knowledge and open data, has long argued that “the best thing to do with your data will be thought of by someone else” \cite{pollock_open_2011}. However, new and creative uses of data, and connections between data, are essential to research and new discoveries. In this paper, we share our experience working with stable unstable data in novel ways using the Princeton Prosody Archive (PPA) as a case study, with a particular focus on HathiTrust Digital Library. We use this as a way to explore the challenges of humanities data and to test the limits of current guidelines and  recommendations for reproducible research, humanities data publication, and consider the implications for the field of computational humanities.


\subsection{Details} \label{sec:intro_details}

You may also include subsections if they help organize your text, but they
are not required. Use as many sections and subsections with whatever names work
for your submission!

\paragraph{Another tip.} In some cases, it may be helpful to use \texttt{paragraph} to title individual paragraphs. For example, if a section describes features for a classifier, you can optionally title each paragraph with the name of each feature. 

\section{Elements}

\subsection{Citing elements}

Here are some examples of how to construct and reference common elements in LaTeX. References to elements such as tables, figures, equations and sections make use of \texttt{label} names that you set. References to citations should use the labels you indicate in \texttt{bibliography.bib}. Change all of these examples and values with your own data. 

We can cite Table~\ref{tab:example} as well as Figure~\ref{fig:example}, and we also cite an example paper \cite{tettoni2024discoverability}.
We can also include mathematical notations, such as:
\begin{align}
f(y) &= x^2. \label{fig:squared}
\end{align}
The line number of the equation can be cited as
Equation~\ref{fig:squared}. You can also cite multiple papers together \cite{barré2024latent, levenson2024textual, bambaci2024steps}, and reference figures or tables indirectly in parentheses (Figure~\ref{fig:example_bigger}). You can also cite other sections or subsections of your paper, such as \S\ref{sec:intro_details}. 


\begin{table}[h]
  \centering 
  \begin{tabular}{cc}
    \toprule
    Column Name 1 & Column Name 2\\
    \midrule
    d1 & d2 \\
    d1 & d2 \\
    d1 & d2 \\
    \bottomrule
  \end{tabular}
  \caption{Example table and table caption.}
  \label{tab:example}
\end{table}


\subsection{Required specifications}

Tables and figures should \textit{not} appear at the top of the first page above the paper title and abstract, but can be placed within the main text, as exemplified by Table~\ref{tab:example}. They may also be placed at the top of non-first pages, as exemplified by Figures~\ref{fig:example} and \ref{fig:example_bigger}. Figures and tables discussed in the main text should appear \textit{before} the References section. Supplementary materials should be referenced by their relevant Appendix section, such as Appendix~\ref{appdx:first}. 

Do \textit{not} change the font size of table and figure captions, or the spacing between text lines, section/subsection titles, tables, figures, and captions. You should size your figures and tables so that they stay within the \texttt{linewidth} of the paper. 

\begin{figure}[t!]
  \centering
  \includegraphics[width=0.4\linewidth]{640x480.png}
  \caption{Example figure and figure caption.}
  \label{fig:example}
\end{figure}

\begin{figure}[t!]
  \centering
  \includegraphics[width=0.4\linewidth]{640x480.png}
  \includegraphics[width=0.4\linewidth]{640x480.png}
  \caption{Example figure, where two \texttt{.png} are placed side by side.}
  \label{fig:example_bigger}
\end{figure}

\section*{Acknowledgements}

This unnumbered section should be blank when submitting your paper. After review, you may include lists of people and organizations who supported the work.

% Print the biblography at the end. Keep this line after the main text of your paper, and before an appendix. 
\printbibliography

% You can include an appendix using the following command
\appendix

\section{HathiTrust Statement for Dataset Distribution} \label{appdx:first}

By my signature, I acknowledge and confirm the following:

\begin{enumerate}
    \item  I am receiving texts from the University of Michigan that are made available under an agreement between my sponsoring institution - [indicate sponsoring institution, e.g., Dartmouth College] - and Google.
    \item  I have read this agreement and agree to abide by its terms and to use the texts in accordance with the statement of my research, as submitted to the University of Michigan.
    \item  I agree to notify the University of Michigan of any changes that are made in the scope or nature of my research.
    \item  I understand that volumes I receive from the University of Michigan may be determined at a later date to be in copyright. I agree to delete these volumes and any copies that have been made upon notification from the University of Michigan. I agree to notify the University of Michigan at feedback@issues.hathitrust.org to confirm deletion of any such volumes.
\end{enumerate}

\rule{\textwidth}{0.5pt}

Name \hspace{0.3\textwidth} Signature \hspace{0.3\textwidth} Date

\rule{\textwidth}{0.5pt}

Title

\rule{\textwidth}{0.5pt}

Email \hspace{0.3\textwidth}  Phone


\section{Example HathiTrust deletion email for public domain dataset} \label{appdx:second}

\begin{verbatim}
Subject: Delete notifications for ht_text_pd dataset
From: HathiTrust <support@hathitrust.org>

Dear HathiTrust dataset recipient,

This email is to notify you that volumes in the HathiTrust
"ht_text_pd" dataset, of which you have downloaded all
or a subset of files, no longer meet the criteria for
inclusion in the dataset, and you no longer are allowed to
use them in your research.

Please review the data you have synced from HathiTrust to
check whether you have the volumes listed below. If so,
delete all copies you retain of these volumes in
accordance with our terms of use. Alternatively, you may
delete your copy of the dataset and re-sync to the updated
dataset.

If you no longer possess HathiTrust datasets, or if you
have other questions regarding datasets, then please email
support@hathitrust.org.

Thank you,

HathiTrust

===BEGIN ID LIST===
[ids omitted]
...
...
...
===END ID LIST===

\end{verbatim}
\end{document}
